%%%%%%%%%%%%%%%%%%%%%%%%%%%%%%%%%%%%%%%%%%%%%%%%%%%%%%
\documentclass[12pt,a4paper]{report}
% o article, book, ...



%%%%%%%%%%%%%%%%%%%%%%%%%%%%%%%%%%%%%%%%%%%%%%%%%%%%%%
% packages...
\usepackage[utf8]{inputenc}
\usepackage[english,italian]{babel}
\usepackage[hyphens]{url}

% Per generare il file PDF aderente alle specifiche PDF/A-1b. Verificarne poi la validità.
%\usepackage[a-1b]{pdfx}

\usepackage{hyperref}
\usepackage{refcheck}
\usepackage{graphicx}
\graphicspath{{./img/}}
\usepackage{subfig}
\usepackage{circuitikz}

\usepackage{underscore} % !!!
\usepackage{hyperref}
%\usepackage{bera}% optional: just to have a nice mono-spaced font
\usepackage{listings}
\usepackage{xcolor}

\colorlet{punct}{red!60!black}
\definecolor{background}{HTML}{EEEEEE}
\definecolor{delim}{RGB}{20,105,176}
\colorlet{numb}{magenta!60!black}

\lstdefinelanguage{json}{
    basicstyle=\normalfont\ttfamily,
    numbers=left,
    numberstyle=\scriptsize,
    stepnumber=1,
    numbersep=8pt,
    showstringspaces=false,
    breaklines=true,
    frame=lines,
    backgroundcolor=\color{background},
    literate=
     *{0}{{{\color{numb}0}}}{1}
      {1}{{{\color{numb}1}}}{1}
      {2}{{{\color{numb}2}}}{1}
      {3}{{{\color{numb}3}}}{1}
      {4}{{{\color{numb}4}}}{1}
      {5}{{{\color{numb}5}}}{1}
      {6}{{{\color{numb}6}}}{1}
      {7}{{{\color{numb}7}}}{1}
      {8}{{{\color{numb}8}}}{1}
      {9}{{{\color{numb}9}}}{1}
      {:}{{{\color{punct}{:}}}}{1}
      {,}{{{\color{punct}{,}}}}{1}
      {\{}{{{\color{delim}{\{}}}}{1}
      {\}}{{{\color{delim}{\}}}}}{1}
      {[}{{{\color{delim}{[}}}}{1}
      {]}{{{\color{delim}{]}}}}{1},
}


% Per inserire testo a caso in attesa di realizzare i capitoli
%\usepackage{lipsum}

\usepackage{todonotes}
%\usepackage[disable]{todonotes} % alla fine mettere

\usepackage[a4paper,top=2cm,bottom=2cm,outer=5cm,verbose,headheight=1cm,heightrounded]{geometry}
\setlength{\marginparwidth}{4.5cm} %per farci stare todonotes, nel final si può rimuovere



%%%%%%%%%%%%%%%%%%%%%%%%%%%%%%%%%%%%%%%%%%%%%%%%%%%%%
\begin{document}

% Frontespizio
\begin{titlepage}
  \begin{center}
    \includegraphics[width=\textwidth]{Logo.jpg}\\
    {\large{\bf Corso di Laurea in Informatica}}
  \end{center}
  \vspace{12mm}
  \begin{center}
    {\huge{\bf CONVERSIONE DI}}\\
    \vspace{4mm}
    {\huge{\bf STRUMENTI VINTAGE}}\\
    \vspace{4mm}
    {\huge{\bf PER LA DATA}}\\
    \vspace{4mm}
    {\huge{\bf PHYSICALIZATION}}\\
  \end{center}
  \vspace{12mm}
  \begin{flushleft}
    {\large{\bf Relatore:}}
    {\large{Andrea Trentini}}\\
    %\vspace{4mm}
    %{\large{\bf Correlatore:}}
    %{\large{...}}\\
  \end{flushleft}
  \vspace{12mm}
  \begin{flushright}
    {\large{\bf Tesi di Laurea di:}}
    {\large{Davide Busolin}}\\
    {\large{\bf Matr. 930814}}\\
  \end{flushright}
  \vspace{4mm}
  \begin{center}
    {\large{\bf Anno Accademico 2020/2021}}
  \end{center}
\end{titlepage}

\listoftodos
\tableofcontents


% REPORT
%\part
%	\chapter
%		\section
%			\subsection
%				\paragraph
%					\subparagraph

% o sections (dipende dal documentclass)
\chapter{Introduzione}
\section{Data Physicalization: definizione}
Una data physicalization (in italiano: fisicalizzazione dei dati) è un artefatto fisico la cui geometria o proprietà materiali codificano dei dati.\cite{dataphysorg:terminology}
%\todo{atrent: in generale i link web li metterei in footnote, se vuoi lasciarli in bib devi mostrare anche l'url}
%\todo{atrent: puoi eliminare tutti i doppi backslash, non servono}
Gli stessi autori della precedente definizione distinguono tra data physicalization come artefatti (per i quali vale quanto scritto sopra),
data physicalization come il processo di dare forma fisica ai dati e data physicalization come l'area di ricerca che unisce da visualizzazione
dei dati e interfacce utenti tangibili.

\section{Visualizzazione dei dati}
Lo scopo della disciplina della visualizzazione dei dati è quello di renderli più comprensibili per gli esseri umani. I display dei computer
sono l'esempio più classico, hanno molti vantaggi grazie alla loro versatilità tra cui quello di fornire rappresentazioni visuali dei dati in
maniera dettagliata e modificabile dinamicamente. Tuttavia, è emersa un'area di ricerca che pone la questione di non limitarsi a una
matrice di pixel per visualizzazione di e interazione con i dati ma sfruttare altri tipi di interazione più naturale per gli esseri umani, come
è stato dimostrato dalla ricerca in Tangible Computing. \cite{hal:dataphys}
\todo{atrent: in generale van riportati i principali concetti esposti negli articoli/libri che citi}

\section{Esempi di data physicalization e visualization}
%\todo{atrent: fai anche esempi non miei, prendi quelli che ti piacciono di più, due o tre esempi}

\subsection{Keyboard Frequency Sculpture}
Un diagramma a barre 3D appoggiato su una tastiera che mostra la frequenza di ciascuna lettera dell'alfabeto. \cite{physlist}, Figura~\ref{fig:keyboardfreq}
\begin{figure}[h]
  \centering
  \includegraphics[width=0.6\textwidth]{keyboardfreq}
  \caption{Keyboard Frequency Sculpture}
  \label{fig:keyboardfreq}
\end{figure}

\subsection{Living Map: Precipitation Visualized with Moss}
Mappa "vivente" che mostra il cambiamento nella quantità di precipitazioni nelle varie zone d'Europa a seguito del riscaldamento climatico
mediante l'erogazione controllata di acqua su un particolare tipo di muschio.  \cite{physlist}, Figura~\ref{fig:livingmap}
\begin{figure}[h]
  \centering
  \includegraphics[width=0.8\textwidth]{livingmap}
  \caption{Living Map: Precipitation Visualized with Moss}
  \label{fig:livingmap}
\end{figure}

\subsection{Self Knitted Scarf of Train Delays}
Sciarpa lavorata a maglia da una pendolare di Monaco di Baviera con sezioni di colore diverso in base ai ritardi giornalieri. \cite{physlist}, Figura~\ref{fig:scarfdelays}.
La grossa sezione rossa è dovuta a un periodo di cantieri ferroviari in cui il treno era sostituito da un autobus.

\begin{figure}[h]
  \centering
  \includegraphics[width=0.8\textwidth]{scarfdelays}
  \caption{Self Knitted Scarf of Train Delays}
  \label{fig:scarfdelays}
\end{figure}

\subsection{Physiradio}
Una radio vintage che riproduce musica di mood/genere correlato alle condizioni metereologiche. \cite{physiradio}, Figura~\ref{fig:physiradio}
\begin{figure}[h]
  \centering
  \includegraphics[width=0.8\textwidth]{physiradio}
  \caption{Physiradio}
  \label{fig:physiradio}
\end{figure}

\subsection{BiblioVisualizer}
Visualizzazione dei dati sul prestito libri nel consorzio Culture Socialità Biblioteche Network Operativo (CSBNO) tramite barra di LED.
\cite{bibliovisgitlab}, Figura~\ref{fig:bibliovisualizer}
\begin{figure}[h]
  \centering
  \includegraphics[width=0.5\textwidth]{bibliovisualizer}
  \caption{BiblioVisualizer}
  \label{fig:bibliovisualizer}
\end{figure}


\todo{busolin: Sezione IoT?}

\section{Obiettivo}
L'obiettivo dell'attività svolta durante il tirocinio è quello di analizzare e sperimentare una forma di fisicalizzazione
dei dati mediante strumenti di misura vintage e dispositivi IoT che non richieda l'utilizzo di computer e monitor e sia di semplice fruizione
da parte di chi non è avvezzo alla tecnologia e può essere intimorito da complicate dashboard o semplicemente privo della volontà
di interagire con dispositivi tecnologici per l'ottenimento di certe informazioni. L'utilizzo di strumenti vintage consente infatti di ottenere
oggetti discreti che possono far parte dell'arredamento domestico, ad esempio su una mensola, i quali invece di essere puramente
decorativi comunicano informazioni utili.

\todo{busolin:  Tecnofobia, Esempi dashboard}


\chapter{Panoramica storica e caratteristiche fisiche degli strumenti}
%\todo{Sezione panoramica storica più caratteristiche fisiche, "istanze"}

\section{Strumenti}
\subsection{Strumenti ad ago}
Sono il tipo più comune di strumento di misura vintage. Un esempio sono i voltmetri / amperometri / multimetri analogici che si utilizzavano
fino a pochi anni fa. Possono essere utilizzati per visualizzare grandezze lineari, facendo attenzione alla scala scelta.
Panoramica in Figura~\ref{fig:strumentiago}.
\begin{figure}[h]
  \centering
  \includegraphics[width=0.9\textwidth]{strumentiago}
  \caption{Assortimento di strumenti ad ago in bachelite, legno e metallo}
  \label{fig:strumentiago}
\end{figure}

\paragraph{Funzionamento}
%generica per ora
Quasi tutti gli strumenti ad ago sono basati su milliamperometri, aventi una resistenza interna, integrati con resistenze di taratura
in serie e/o in parallelo in base all'utilizzo finale desiderato. Per ottenere un voltmetro con fondoscala adeguato, è spesso necessario
inserire una resistenza in serie con il milliamperometro, in modo da ridurne la sensibilità.
Questo principio si basa sulla legge di Ohm: conoscendo il flusso di corrente necessario normalmente per arrivare a fondo scala, è
possibile calcolare la resistenza totale necessaria per portare l'ago al 100\% in corrispondenza di una tensione desiderata.
Esempio di scheda resistenze in Figura~\ref{fig:schedaresistenze}.

\begin{figure}[h]
  \centering
  \includegraphics[width=0.8\textwidth]{schedaresistenze}
  \caption{Scheda resistenze di un voltmetro analogico}
  \label{fig:schedaresistenze}
\end{figure}


\subsection{Dekatron}
Valvola termoionica che svolge la funzione di dispositivo di conteggio, molto usata all'interno dei computer negli anni '50 e '60 e ora
relegata all'utilizzo hobbistico, ad esempio per la realizzazione di particolari orologi. \cite{itwiki:118300354}, Figura~\ref{fig:dekatron}.

\begin{figure}[h]
  \centering
  \includegraphics[width=0.9\textwidth]{dekatron}
  \caption{Dekatron \cite{itwiki:118300354}}
  \label{fig:dekatron}
\end{figure}

\subsection{Occhio magico}
Valvola termoionica usata per rappresentare l'intensità di un segnale elettrico. Storicamente impiegata nelle radio come
indicatore di sintonia. \cite{itwiki:119250611}, Figura~\ref{fig:occhiomagicofasi}.

\begin{figure}[h]
  \centering
  \includegraphics[width=0.9\textwidth]{occhiomagicofasi}
  \caption{Occhio magico in diverse fasi di illuminazione \cite{itwiki:119250611}}
  \label{fig:occhiomagicofasi}
\end{figure}

\subsection{Oscilloscopio}
Strumento utilizzato per visualizzare l'andamento nel tempo di un segnale in maniera grafica \cite{itwiki:120862120}. Se pilotato a dovere è in grado di
visualizzare delle grafiche. Esempio in Figura~\ref{fig:oscilloscopio}.

\begin{figure}[h]
  \centering
  \includegraphics[width=0.8\textwidth]{oscilloscopio}
  \caption{Oscilloscopio impiegato nella visualizzazione di un disegno \cite{instructablesoscilloscope}}
  \label{fig:oscilloscopio}
\end{figure}

\section{Componenti}
\subsection{Interruttori / Pulsanti}
Formati da un semplice meccanismo che permette di aprire o chiudere un circuito.
\subsection{Commutatori a n posizioni}
Sono utilizzati per chiudere uno tra n possibili circuiti attraverso la loro rotazione.
\subsection{Potenziometri}
Consentono di variare una differenza di potenziale nel continuo mediante un partitore resistivo ``mobile".
\subsection{Morsetti / Connettori}
Consentono di collegare e scollegare cavi o sonde allo strumento.
\subsection{Lampadine}
Fonte luminosa. Generalmente quelle che si possono trovare nei vecchi apparati sono piccole e a incandescenza. Contengono un filamento
metallico, generalmente di tungsteno, che scaldandosi emette della luce.
\subsection{Buzzer/Altoparlanti}
Consentono di produrre dei suoni, un esempio di applicazione è nei tester di continuità elettrica, che emettono un ``beep" alla rilevazione.
\subsection{Termometri}
Componente in grado di misurare la temperatura ambientale. Negli strumenti di misura possono essere usati per compensare le misurazioni
di componenti sensibili alle variazioni di temperatura nell'ambiente di utilizzo.
\subsection{Inclinometri a mercurio}
Particolari interruttori basati sulla gravità, i più semplici contengono una goccia di mercurio e due contatti elettrici, la cui chiusura dipende dalla
direzione della forza di gravità a cui è soggetto il componente: se la goccia viene spinta verso i contatti il circuito è chiuso, altrimenti rimane aperto.


\chapter{Generalizzazione degli strumenti di misura}
%\todo{Sezione mapping componenti e interventi di modifica possibili}
In questo capitolo si tenterà di generalizzare le caratteristiche dei componenti degli strumenti di misura ai fini di associarle a componenti
software e tipi di dato.
%\todo{atrent: strada giusta! fai anche esempi pratici finali}

\section{Interruttori / Pulsanti}
Possono essere mappati a valori booleani (true/false). Collegabili da un capo a GND e dall'altro capo a un GPIO del microcontrollore in modalità
input pullup. Possono essere utilizzati nel codice mediante la funzione digitalRead o come switch component su esphome. Esempio di collegamento
in Figura~\ref{fig:switchconnection}

\begin{figure}[h]
  \centering
  \begin{circuitikz} \draw
    (0,0) to[cute open switch] (0,4)
   -- (4,4) node[above]{GPIO}
    to[twoport,t={ESP}] (4,0) node[below]{GND}
    -- (0,0)
    ;
  \end{circuitikz}
  \caption{Schema collegamento switch all'ESP}
  \label{fig:switchconnection}
\end{figure}


\section{Commutatori a n posizioni}
Possono essere mappati in un programma a tipi \texttt{enum}, utilizzabili per selezionare tra un insieme di modalità o un insieme di valori
utilizzabili come parametro in un programma. Ad esempio un commutatore a 7 posti può essere utilizzato per selezionare un giorno della
settimana.

Per evitare di occupare tanti pin del microcontrollore quante sono le possibili posizioni di un commutatore, è possibile convertirlo in un
partitore di tensione mediante l'ausilio di resistenze tutte uguali collegate come da esempio in Figura~\ref{fig:commutatoreprimadopo},
ottenendo una sorta di potenziometro che invece di variare nel continuo assume valori in step regolari tra il minimo e il massimo
dell'ingresso, che possono essere agevolmente distinti e interpretati come valori discreti.
Esempio di collegamento in figura~\ref{fig:rotaryconnection}. Per la lettura valgono le considerazioni fatte sui potenziometri, bisognerà
poi via software fare dei calcoli per riconoscere gli step e utilizzarli a dovere nel codice.

\begin{figure}[h]
  \centering
  \subfloat[Prima]{{\includegraphics[height=3.5cm]{commutatoreStock}}}
  \enspace
  \subfloat[Dopo]{{\includegraphics[height=3.5cm]{commutatorePartitore}}}
  \caption{Dettaglio commutatore prima e dopo la modifica come partitore}
  \label{fig:commutatoreprimadopo}
\end{figure}

\begin{figure}[h]
  \centering
  \begin{circuitikz} \draw
    (0,0) to[twoport,t={ESP},  n=esp] (0,6)

    (1.5,3)  node[rotary switch  -=4 in 90 wiper 15, anchor=in](SW){}
    (SW.in) -- (esp)
   

    (4,0) to[R] (4,2) to[R] (4,4) to[R] (4,6) 
    (0,6) -- (4,6)
    (0,0) -- (4,0)

    (SW.out 1) to[short, -*] (4,6) node[above]{3.3V}
    (SW.out 2) to[short, -*] (4,4)
    (SW.out 3) to[short, -*] (4,2)
    (SW.out 4) to[short, -*] (4,0)  node[below]{GND}
  ;
  \end{circuitikz}
  \caption{Schema collegamento commutatore modificato all'ESP}
  \label{fig:rotaryconnection}
\end{figure}

\section{Potenziometri}
Possono essere mappati a valori aventi intervallo arbitrario, con una precisione limitata dalla risoluzione del convertitore analogico-digitale
integrato nel microcontrollore. Per poter leggere l'output è necessario collegarli a un GPIO che supporta la lettura di segnali analogici
e nel codice utilizzare la funzione analogRead. Su esphome è necessario usare sensori di tipo adc.
Esempio di collegamento in Figura~\ref{fig:potconnection}

\begin{figure}[h]
  \centering
  \begin{circuitikz} \draw
    (0,4) to[potentiometer, n=pot] (0,0)
    (0,4) to[short] (4,4) node[above]{3.3V}
    to[twoport,t={ESP}, n=esp] (4,0) node[below]{GND}
    -- (0,0)
    (pot.wiper) -- (esp)
  ;
  \end{circuitikz}
  \caption{Schema collegamento potenziometro all'ESP}
  \label{fig:potconnection}
\end{figure}

\section{Connettori multipli}
Alcuni strumenti dispongono di più ingressi per le sonde di misura, adatti a connettori a banana o non standard, specifici alle
diverse modalità di misura e/o i diversi fondoscala tra cui scegliere. Possono essere utilizzati come selettori a livello software in base a
quali degli ingressi vengono ponticellati, in maniera simile alle connessioni tra utenze telefoniche nei vecchi centralini o agli spinotti per
gli scambi di lettere presenti sulla macchina Enigma. Si riporta un esempio in Figura~\ref{fig:connettorimultipli}

\begin{figure}[h]
  \centering
  \includegraphics[width=0.8\textwidth]{connettorimultipli}
  \caption{Amperometro con connettori separati in base alla scala}
  \label{fig:connettorimultipli}
\end{figure}

\subsection{Lampadine}
Possono essere utilizzate per rappresentare uno stato binario o comunicare diverse informazioni mediante dei pattern di lampeggiamento.
In genere con i sistemi embedded si adoperano i LED al posto delle lampadine a incandescenza, in quanto possono essere pilotati
direttamente dai pin del microcontrollore, salvo eventuali resistenze in serie per ridurre il flusso di corrente.
Nel codice di Arduino un LED può essere pilotato mediante una digitalWrite per funzionamento on/off o una analogWrite per
intensità variabili mediante PWM (in maniera simile a come si è fatto con l'ago dello strumento per il prototipo). In esphome si può
utilizzare un componente di tipo light.

\subsection{Buzzer/Altoparlanti}
Possono essere usati per dare indicazioni audio, ad esempio in caso di errore. I buzzer possono solitamente essere pilotati direttamente
dal microcontrollore, mentre gli altoparlanti spesso richiedono un amplificatore di potenza.
Il funzionamento dal punto di vista software dipende dal tipo di buzzer utilizzato: un buzzer attivo contiene al suo interno un oscillatore
e quindi genererà un beep mentre è alimentato. Un buzzer passivo si comporta in maniera simile a un altoparlante in quanto ha bisogno di
un segnale audio in input. Attraverso esphome è possibile riprodurre delle semplici melodie con buzzer passivi utilizzando il componente rtttl.

\subsection{Termometri} \todo{busolin: Se anche uno strumento ha al suo interno un termometro di taratura, quasi sicuramente non sarà integrabile con l'ESP. Menziono lo stesso la possibilità di aggiungere un DHT?}

\subsection{Inclinometri a mercurio}
Ai fini implementativi hanno un funzionamento al pari degli interruttori/pulsanti. L'unica differenza sta nel principio fisico di funzionamento
e quindi al significato che si può dare all'output nel codice.

\chapter{Guida alla trasformazione degli strumenti vintage}
\todo{Sezione guida alla conversione vera e propria che fa riferimenti alla sezione precedente}

%\todo{atrent: manca una trattazione di come è fatto e come funziona lo strumento ad ago (che sarà il più frequente oggetto di visualizzazione), spiega bene la sua architettura, wikipedia is your friend}


%\todo{atrent: tenterei anche una panoramica storica degli strumenti di ``visualizzazione'' (occhi magici, elettrofori, dekatron, oscilloscopi, ...) dando idee su come si potrebbero usare per datafisicalizzare}

\section{Introduzione}
Una volta appurate le caratteristiche degli strumenti, è opportuno discutere degli interventi fisici necessari alla loro conversione.


\section{Interventi}
\subsection{Intervento minimale}
Il metodo più semplice per ottenere un oggetto funzionante è quello di collegare il GPIO del microcontrollore settato come output
e GND rispettivamente al terminale positivo e negativo dello strumento di misura in modalità voltmetro. Se la tensione in uscita massima
coincide con il fondoscala dello strumento di misura questo è sufficiente, altrimenti è necessario ridurla via software oppure
utilizzare un partitore di tensione o un potenziometro. Nel caso in cui la tensione in uscita massima del microcontrollore sia inferiore
ad ogni possibile fondoscala dello strumento di misura, questo metodo non è attuabile. Esempio di intervento minimale in
Figura~\ref{fig:interventominimale}. Notare il cavo collegato ai terminali: il microcontrollore è esterno
e i commutatori devono rimanere nella posizione rappresentata per consentire un corretto funzionamento.

\begin{figure}[h]
  \centering
  \includegraphics[width=0.8\textwidth]{interventominimale}
  \caption{Esempio di intervento minimale}
  \label{fig:interventominimale}
\end{figure}

\subsection{Intervento approfondito}
Un metodo più invasivo ma più pulito consiste nel modificare la struttura interna dello strumento di misura, collegandosi ai capi dello
strumento interno vero e proprio bypassando selettore di modalità/fondoscala.
Se non si vuole rimpiazzare la resistenza interna valgono le stesse osservazioni fatte poc'anzi sui fondoscala: è necessario che
esista una modalità con fondoscala minore o uguale alla tensione di uscita massima del microcontrollore, altrimenti sarà necessario
utilizzare una resistenza più piccola di quella/e già presente/i. 

Esiste però un caso in cui, nonostante la rimozione delle resistenze di taratura, la sensibilità dello strumento rimane più bassa di quanto
desiderato e la tensione massima che può emettere il microcontrollore non basta per arrivare a fondoscala. In questo caso le alternative
sarebbero adoperare un survoltore per raggiungere la tensione necessaria oppure rinunciare e provare con un altro strumento.

Questa operazione consente di integrare (a condizione che ci sia spazio, gli strumenti in Figura~\ref{fig:strumentiago} sono quasi tutti
adatti) il microcontrollore all'interno dello strumento in modo da ottenere un sistema composto da un solo oggetto.
Se si ha la facoltà di inserire una batteria diventa anche incredibilmente discreto e di più facile posizionamento, anche dove è
scomodo portare una fonte di alimentazione.

\subsection{Interventi aggiuntivi}
Ulteriori possibili modifiche meno banali includono l'aggiunta di elementi decorativi quali delle luci o funzionali quali bottoni o altoparlanti
per ampliare le possibilità di utilizzo. Un'altra possibilità di modifica ove possibile è la sostituzione dell'etichetta dietro all'ago per adattare
la scala rappresentata alla vera scala dei dati (a condizione che questa rimanga fissa). Adoperando un display e-paper si potrebbe
adattare dinamicamente la scala rappresentata dallo strumento senza più bisogno di interventi fisici. Esempio di strumento dopo
modifiche avanzate in Figura~\ref{fig:interventocompleto}: è stato sostituito il terminale sinistro con un connettore USB da pannello
ai fini di interfacciarsi con il microcontrollore e alimentarlo. Il terminale destro è stato sostituito con un indicatore LED, il commutatore
destro con un pulsante e all'interno è stato posizionato un buzzer che consente di riprodurre dei suoni.

\begin{figure}[h]
  \centering
  \includegraphics[width=0.9\textwidth]{interventocompleto}
  \caption{Esempio di strumento dopo modifiche avanzate}
  \label{fig:interventocompleto}
\end{figure}

\chapter{Il prototipo - IoTDial}
Questo capitolo tratterà dell'implementazione delle varie feature del prototipo realizzato durante il tirocinio, chiamato IoTDial,
il loro utilizzo e alcune osservazioni al riguardo.

\section{Hardware}
Il prototipo era inizialmente molto semplice dal punto di vista hardware: consisteva in un voltmetro analogico ad ago impostato con
fondoscala a 3V e un WeMos D1 mini basato su ESP8266. Il terminale positivo del voltmetro era connesso a un GPIO dell'ESP,
mentre il terminale negativo a GND.

Successivamente sono stati eseguiti maggiori interventi fisici fino ad arrivare alla situazione visibile in Figura~\ref{fig:interventocompleto}
e descritta nella corrispondente sezione, con il microcontrollore inserito all'interno della scocca dello strumento e alcune modifiche estetiche
e funzionali.

\section{Software}
La parte software del prototipo è stata realizzata utilizzando il linguaggio di programmazione di Arduino, esteso all'utilizzo con
ESP8266 e con l'aiuto di numerose librerie. Come IDE è stato usato Visual Studio Code con l'estensione PlatformIO, la quale
permette di specificare le dipendenze direttamente in file di configurazione e di scaricare automaticamente le librerie necessarie
per ogni progetto.

\subsection{PWM}
Il principio di base che permette di produrre una tensione variabile da parte di un sistema digitale è la Pulse-Width Modulation.
Ciò permette di controllare finemente la posizione dell'ago del voltmetro via software.

\subsection{Richieste API}
Il primo passo verso la fisicalizzazione dei dati è stato quello di ottenerli, mediante richieste ad API REST in locale o su Internet.
Per questo è stata utilizzata la libreria \texttt{ESP8266HTTPClient}.

\lstset{language=C++,
                basicstyle=\ttfamily,
                breaklines=true,
                keywordstyle=\color{blue}\ttfamily,
                stringstyle=\color{red}\ttfamily,
                commentstyle=\color{green}\ttfamily,
                morecomment=[l][\color{magenta}]{\#}
}
\begin{lstlisting}
bool begin;
if (use_https) {
  begin = http.begin(ssl_client, apiUrl);
  Serial.print("[HTTPS] ");
} else {
  begin = http.begin(client, apiUrl);
  Serial.print("[HTTP] ");
}

Serial.print("begin... ");
Serial.println("URL: " + apiUrl);
\end{lstlisting}

Le richieste HTTP hanno funzionato fin da subito, per HTTPS c'è stato più da lavorare in quanto non erano eccessivamente
documentate e tutte le soluzioni trovate online prevedevano l'utilizzo di stringhe per memorizzare intere risposte, che per via
della limitata disponibilità di memoria dell'ESP non è sempre una soluzione adeguata.
Un'alternativa fornita dalla libreria è quella di utilizzo di uno stream, per poter leggere dalla risposta un byte alla volta senza doverla
memorizzare per intero, potendo ignorare le parti non necessarie. Questo meccanismo però è di facile utilizzo solo finché ci si limita a
HTTP. Per HTTPS la funzione per estrarre lo stream della risposta, secondo la documentazione della libreria, non è supportata.
Per questo motivo è stato utilizzato come stream direttamente il BearSSL::WiFiClientSecure referenziato dalla connessione, impostato in
modalità insicura (senza la verifica della fingerprint), che però non sempre presenta un output "pulito" della risposta ma essa potrebbe
essere preceduta da una linea contentente (presumibilmente) la lunghezza in byte della risposta, che va eliminata prima di poter
passare lo stream al parser JSON.

Si riporta la sezione di codice che chiama un metodo di callback che accetta uno stream per argomento. Notare che per una richiesta
HTTP è utilizzato il metodo getStream messo a disposizione dalla libreria, mentre per HTTPS viene passato direttamente il client.
\begin{lstlisting}
if (httpCode == HTTP_CODE_OK) {
  if (use_https) {
    callback(ssl_client);
  } else {
    callback(http.getStream());
  }
}
\end{lstlisting}

\subsection{Risposte JSON}
La quasi totalità delle API mette a disposizione risposte contenenti dati in formato JSON (JavaScript Object Notation). È stata quindi
utilizzata la libreria \texttt{ArduinoJson} per la deserializzazione e la scelta del campo da estrarre.
JSON è stato successivamente usato anche come formato per la serializzazione e deserializzazione delle impostazioni da salvare sulla
memoria flash dell'ESP per permetterne la persistenza anche a seguito di perdite di tensione.

Segue una porzione di codice della callback che evidenzia l'utilizzo di uno stream con buffer da 64 byte ai fini di un miglioramento delle
performance, la deserializzazione del filtro e la "pulizia" dello stream nel caso in cui si usi HTTPS ed esso contenga caratteri iniziali
diversi dalle parentesi.
\begin{lstlisting}
void stream_callback(Stream &stream) {
  ReadBufferingStream rbs(stream, 64);
  StaticJsonDocument<200> filter;
  deserializeJson(filter, filterJSON);
  Serial.println("--- Start of text excluded from stream ---");
  while (rbs.available() && char(rbs.peek()) != '{' && char(rbs.peek()) != '[') {
    Serial.print(char(rbs.read()));
  }
  Serial.println("---  End of text excluded from stream  ---");
  DynamicJsonDocument doc(2000);
  deserializeJson(doc, rbs, DeserializationOption::Filter(filter));
\end{lstlisting}

\subsection{Approfondimento su filtro JSON e path}
Il centro del funzionamento della modalità API e la sua adattabilità all'utilizzo di fonti dati diverse risiede nei campi
filterJSON e path.
\texttt{filterJSON} consente di evitare che tutta la risposta delle API venga salvata in memoria. Viene quindi scartato ogni campo che non
appaia come placeholder all'interno della stringa, ad esempio:
\texttt{"\{Lines: [\{WaitMessage: true\}]\}"} consente di mantenere
in memoria solo il campo con chiave WaitMessage di ogni elemento contenuto nell'array avente chiave Lines. Questa funzionalità è
implementata e ben documentata nella libreria \texttt{ArduinoJSON}.~\cite{arduinojsonfiltering}

Questo filtro non è sufficiente come indicatore del campo desiderato, in quanto il documento JSON salvato in memoria conterrà comunque
tutta la gerarchia e per come funzionano i placeholder sugli array non è neanche detto che ci sia in totale un solo campo con un valore
assegnato.
Per questo motivo è stato aggiunto il campo \texttt{path} alla configurazione e implementato nel codice un metodo di parsing specifico.
Questa stringa, nel programma, viene interpretata token per token prendendo gli slash come separatori e usata per esplorare
la gerarchia del documento JSON. Un token completamente numerico viene visto come indice di un array, altrimenti come chiave da
cercare all'interno di un oggetto.

Per fare un esempio coerente con il precedente: \texttt{"Lines/0/WaitMessage"} verrà approcciato nel seguente modo:
\begin{itemize}
  \item Interpretare la radice come un oggetto, cercare la chiave \texttt{Lines} e spostarvisi all'interno;
  \item Interpretare il successivo livello come un array, estrarre l'elemento in posizione 0 e spostarvisi all'interno;
  \item Interpretare il successivo livello come un oggetto, cercare la chiave \texttt{WaitMessage} e spostarvisi all'interno.
\end{itemize}
A questo punto, finita la stringa, il programma assumerà di trovarsi dinnanzi a un campo interpretabile come numerico (tollera molto,
ad esempio la stringa ``123aaa" verrà interpretata come 123.00) e lo estrarrà come risultato. %excursus JsonVariant?

Questo approccio, grazie alla cura con cui è implementata la libreria \texttt{ArduinoJson}, è molto tollerante a errori o inesattezze.
Se nel JSON non fosse presente un campo desiderato durante il parsing della path, così come se non fosse possibile la conversione
da stringa a numero presente subito dopo, non avverrebbe alcun crash del programma.
È quindi possibile avere filtro e path inizialmente non corretti e adattarli tramite \textit{trial and error} senza ottenere riavvii
indesiderati del microcontrollore a seguito di eccezioni.

Nella seguente porzione di codice è possibile osservare le istruzioni cruciali per l'implementazione del parsing della path:
\begin{lstlisting}
JsonVariant jv = doc.as<JsonVariant>();
token = strtok(path_buf, "/");
while (token != NULL) {
  bool object = false;
  char *p = token;
  while (*p != 0) {
    if (!isDigit(*p)) {
      object = true;
      break;
    }
    p++;
  }
  if (object) {
    jv = jv[token];
  } else {
    jv = jv[atoi(token)];
  }
  token = strtok(NULL, "/");
}
\end{lstlisting}
Un'alternativa possibile sarebbe quella di distinguere chiavi e indici mediante appositi prefissi, evitando la scansione completa di ogni
token per determinare se esso sia numerico o meno. Un ulteriore vantaggio sarebbe quello di poter avere nel documento JSON
chiavi completamente numeriche senza che esse vengano confuse con indici di array.

\subsection{File di configurazione JSON}
La configurazione è salvata nella memoria flash del microcontrollore
in formato JSON, esempio:

\begin{lstlisting}[language=json,firstnumber=1]
{
  "apiUrl": "https://giromilano.atm.it/proxy.ashx",
  "filterJSON": "{Lines: [{WaitMessage: true}]}",
  "post_payload": "url=tpPortal/geodata/pois/stops/12493",
  "path": "Lines/0/WaitMessage",
  "min_value": 0,
  "max_value": 15,
  "min_pwm": 0,
  "max_pwm": 920,
  "request_interval_ms": 15000,
  "mode": 2
}
\end{lstlisting}


\noindent Dove:
\begin{itemize}
  \item \texttt{apiUrl} è l'URL della risorsa (stringa)
  \item \texttt{post_payload} in caso di richiesta POST contiene la stringa da inviare nel body
  \item \texttt{filterJSON} è un documento JSON che contiene \texttt{true} come placeholder del campo che si vuole considerare dalla risposta (stringa/oggetto)
  \item \texttt{path} è una stringa che contiene il "percorso" del campo che si vuole rappresentare fisicamente (campo: float, path: stringa)
  \item \texttt{min_value} indica il valore minimo della scala per rappresentare il valore (float)
  \item \texttt{max_value} indica il valore massimo (float)
  \item \texttt{min_pwm} indica il valore minimo di uscita della PWM da mappare al valore restituito dalle API (int)
  \item \texttt{max_pwm} indica il valore massimo [su ESP il duty cycle massimo equivale a 1023] (int)
  \item \texttt{request_interval_ms} indica l'intervallo di tempo in millisecondi che intercorre tra le richieste alla risorsa (int)
  \item \texttt{mode} indica il codice numerico della modalità di funzionamento da selezionare
\end{itemize}

\subsection{Map}
Una volta ottenuto il valore da rappresentare è necessario trasformarlo in una tensione, ottenuta tramite PWM il cui duty cycle è
proporzionale a un numero a 10 bit (0 - 1023). Per fare questo è possibile utilizzare la funzione \texttt{map} già presente nella
libreria standard di Arduino.

Durante i test è emerso che tale funzione non si comporta particolarmente bene quando uno dei range è formato da piccoli numeri
decimali. Si è quindi preferito inserire nel codice direttamente la formula matematica su cui si basa \texttt{map}:
\[out = \frac{(value - fromLow) \times (toHigh - toLow)}{(fromHigh - fromLow)} + toLow \]

\noindent Dove:\cite{arduinomap}
\begin{itemize}
  \item value: Il numero da mappare.
  \item fromLow: Il limite inferiore dell'intervallo corrente del valore
  \item fromHigh: Il limite superiore dell'intervallo corrente del valore
  \item toLow: Il limite inferiore dell'intervallo di destinazione del valore
  \item toHigh: Il limite superiore dell'intervallo di destinazione del valore
\end{itemize}

\subsection{Task}
Attività periodiche come le richieste API sono state rese dei task per mezzo della libreria \texttt{TaskScheduler}, che permette di
impostare un intervallo di tempo tra esecuzioni ripetute di una procedura.
Segue una piccola selezione dei task presenti nel codice a titolo di esempio della loro dichiarazione e utilizzo.
\begin{lstlisting}
#include <TaskScheduler.h>
Scheduler ts;
void mqtt_loop() {
  mqtt_client.loop();
}
Task mqtt_loop_task(10 * TASK_MILLISECOND, TASK_FOREVER, mqtt_loop);
void ftp_handle() {
  ftp.handleFTP();
}
Task ftp_handle_task(10 * TASK_MILLISECOND, TASK_FOREVER, ftp_handle);
[in setup()]
  ts.addTask(mqtt_loop_task);
  ts.addTask(ftp_handle_task);
  mqtt_loop_task.enable();
  ftp_handle_task.enable();

\end{lstlisting}


\subsection{Configurazione via MQTT}
Come modalità di modifica ``a runtime" della configurazione, è stata implementata la possibilità di inviare messaggi MQTT a diversi
topic, aventi come radice \texttt{IoTDial/<hostname>}, dove l'hostname è formato da \texttt{IoTDial-MACADDR} e MACADDR è
l'indirizzo MAC della scheda di rete del microcontrollore (univoco al mondo) escludendo i due punti. Esempio: 
\texttt{IoTDial/IoTDial-483fda781185} (da qui in poi indicato come \texttt{<root>}.
A \texttt{<root>/setFromJSON} è possibile inviare un intero JSON di configurazione, altrimenti i campi singoli possono essere
settati tramite i topic:
\begin{itemize}
  \item \texttt{<root>/setApiUrl}
  \item \texttt{<root>/setPostPayload}
  \item \texttt{<root>/setFilterJson}
  \item \texttt{<root>/setPath}
  \item \texttt{<root>/setMinValue}
  \item \texttt{<root>/setMaxValue}
  \item \texttt{<root>/setMinPwm}
  \item \texttt{<root>/setMaxPwm}
  \item \texttt{<root>/setRequestIntervalMs}
  \item \texttt{<root>/setMode}
\end{itemize}
\noindent Sono inoltre presenti alcuni topic di utilità:
\begin{itemize}
  \item \texttt{<root>/setConfigFile} imposta il nome del file di configurazione da usare dal momento dell'invio in avanti. Tale
		impostazione viene automaticamente salvata nella memoria flash quindi è persistente al riavvio.
  \item \texttt{<root>/saveToFlash} Salva le impostazioni correnti in memoria flash, sovrascrivendo (o creando) il file di configurazione
		in uso.
\end{itemize}

\subsection{Credenziali WiFi}
Per la connessione alla rete wireless, grazie alla libreria \texttt{WiFiManager}, l'ESP memorizzerà le ultime credenziali utilizzate
per ritentare la connessione al riavvio. In caso le credenziali non siano valide o non ve ne siano di salvate, verrà avviato un
Access Point temporaneo con captive portal (lo stesso che comunemente viene usato su reti pubbliche o aziendali per richiedere
l'autenticazione) al fine di permettere all'utente di inserire delle credenziali valide mediante form web.

\missingfigure{screenshot}

Successivamente viene tentata la connessione. Se non ha successo viene ripetuto lo stesso procedimento, altrimenti viene disattivato
l'Access Point temporaneo, avviene la connesisone e il programma prosegue.

\subsection{Configurazione tramite form web}
Un'altra forma di modifica della configurazione è stata implementata tramite un web server configurato per mezzo della libreria
\texttt{ESPAsyncWebServer} ed esposto alla rete locale all'indirizzo IP o hostname del microcontrollore.
 Inizialmente la pagina conteneva un semplice form tramite il quale era possibile inviare un intero JSON
di configurazione, successivamente è stata aggiunta la possibilità di impostare i singoli parametri attraverso un form precompilato
e richieste AJAX. Un punto debole di questa soluzione è la scarsa flessibilità del web server, che gestisce richieste POST solo codificate
come Form Data, quindi invece che poter mandare un semplice JSON nel body della richiesta come da prassi con AJAX, è stato necessario
trasformarlo in FormData come valore associato alla chiave setFromJSON.
Nel dettaglio, erano esposti i seguenti endpoint:
\begin{itemize}
  \item /: Pagina di configurazione (index.html)
  \item /get/running-conf: Restituisce la configurazione attualmente in esecuzione in formato JSON
  \item /get/saved-conf: Restituisce il file di configurazione presente nella memoria flash
  \item /post: permette di inviare un intero JSON di configurazione (anche incompleto, vengono modificati solo i campi specificati) come
        valore associato alla chiave setFromJSON. È inoltre possibile asssociare alla chiave saveToFlash il valore `true' per copiare
        la configurazione in esecuzione nella memoria flash.
\end{itemize}

Si riporta a seguito in Figura la pagina di configurazione

\missingfigure{screenshot pagina web}

I form erano precompilati con la configurazione in esecuzione (\emph{running-conf}) ed era presente una sezione che riportava la
configurazione salvata nella flash (\emph{saved-conf}).
Le modifiche non erano automaticamente salvate nella memoria flash ma ciò doveva essere richiesto esplicitamente. Nei form di configurazione
era presente una checkbox da spuntare in caso si volesse che questo avvenisse.

Questa feature è successivamente stata disattivata perché l'esecuzione contemporanea di un webserver e di richieste API creava
problemi di allocazione della memoria del microcontrollore che spesso risultavano in crash improvvisi. Oltretutto, la spartana
interfaccia web risultava ridondante rispetto alle modalità di configurazione già presenti in quanto non garantiva una maggiore
semplicità di utilizzo per l'utente, che deve essere comunque esperto per apportare certe modifiche.

\subsection{OTA}
Sono stati implementati gli aggiornamenti OTA (Over the Air), processo tramite il quale è possibile eseguire l'upload di firmware
mediante WiFi invece che tramite porta seriale. Questo rimuove l'obbligo di utilizzo di una connessione cablata ed è molto utile
in situazioni di difficile accesso fisico al microcontrollore.

Il processo di upload usa Digest-MD5 per la verifica dell'integrità, onde evitare di rendere il dispositivo inservibile. È inoltre possibile
migliorare la sicurezza dell'operazione mediante utilizzo di password e hash firmati. \cite{espota}

\section{Ruoli}
Prima di parlare del delle modalità operative di IoTDial è il caso di soffermarsi su chi vi può interagire:
\subsection*{Sviluppatore}
È in grado di apportare modifiche al codice sorgente. Può aggiungere, rimuovere e  modificare funzionalità del programma
e apportare modifiche dirette ai file di configurazione.
\subsection*{Utente esperto}
Conosce le API ed è in grado di comprendere il formato dei dati che restituiscono. Gli è sufficiente un'interfaccia anche spartana
per selezionare la sorgente dei dati e il campo specifico che vuole rappresentato.
\subsection*{Utente inesperto}
Non conosce le API né il formato JSON, è possibile che abbia anche una limitata conoscenza del mondo informatico:
ha bisogno di un prodotto già pronto che richieda la minima configurazione possibile e questa
deve essere particolarmente intuitiva, ad esempio una piccola interfaccia web per scegliere la rete WiFi e inserirne la password
al primo avvio. La sorgente dei dati deve essere preconfigurata e se ne viene resa disponibile più di una la scelta deve essere molto
semplice, possibilmente tramite interazione fisica con il dispositivo.

\todo{atrent: questi sono i ruoli, mancano gli use case veri e propri (parlando formalmente, faccio riferimento a UML)}

\todo{atrent: in generale metterei tanti schemi e disegni}

\section{Modalità operative}

\subsection{Modalità API}
Questa modalità di operazione si basa sull'esecuzione ad intervalli di tempo configurabili di richieste HTTP o HTTPS ad API REST
e sulla conversione del risultato in una tensione di uscita, avendo in più le funzionalità descritte precedentemente nella sezione
sulla realizzazione software.
Gli interventi al codice richiedono uno sviluppatore, la configurazione della fonte API e la gestione dei multipli file di configurazione sono nelle
corde di un utente esperto mentre per l'utente inesperto il massimo della funzionalità potrebbe essere quella di alternare tra fonti di dati o
configurazioni diverse mediante l'interazione con un commutatore fisico.

\subsection{Modalità MQTT / smart home}
\todo{busolin: descrizione vecchia modalità MQTT e esphome}
Questa modalità era stata inizialmente pensata come complemento della modalità API: condivideva in parte i parametri di configurazione
e consisteva nel mappare in uscita sullo strumento, al posto di un valore ritornato dalle API, un valore passato tramite messaggio MQTT
a uno specifico topic (\texttt{<root>/setValue}).

Andando avanti con le idee di sviluppo era in cantiere un'estensione del programma per l'integrazione nello scambio di messaggi di componenti
aggiuntive quali switch, commutatori, ecc. già descritte in altre sezioni. Questo avrebbe voluto dire scrivere codice per ogni tipo di device
aggiuntivo da integrare nel sistema e trovare un modo per abilitare ciò che era necessario in base alla configurazione fisica del dispositivo,
il tutto in ottica di una possibile integrazione con sistemi di domotica per il raccoglimento e l'elaborazione dei dati dei vari sensori e input.

Questa lista di funzionalità rispecchia molto bene quella di esphome, per cui è stato relegato il funzionamento stand-alone del solo ago
con codice scritto apposta solo alla ``modalità API", mentre per un'integrazione con sistemi di domotica e la gestione di vari input e output
sempre diversi si è scelto di fare totale affidamento su esphome, che mediante la sua estensiva documentazione e compatibilità con una vasta
gamma di componenti esterne si presta molto bene all'incarico.

...

In questo caso, uno sviluppatore è richiesto per adattare il file di configurazione di esphome ai vari dispositivi, specialmente per quanto riguarda
automatismi interni indipendenti da sistemi di smart home. Anche in questo caso per l'utente inesperto è necessario fornire un dispositivo già
configurato e pronto all'uso.

\chapter{Conclusioni}

\todo{atrent: da qualche parte mettere i risultati dei questionari}

%\todo{atrent: oltre alla presentazione all'hacklab (domani!) proviamo a organizzare una sessione ad hoc almeno con gli studenti del corso SistEmbed (e magari quelli di PROS?)}

\section{Reperibilità delle API}
Durante la ricerca di API rappresentabili mediante Data Physicalization è risultato evidente che molte di quelle più congeniali non
sono pubbliche e spesso sono anche di difficile utilizzo. L'esperienza personale è stata con quelle di ATM per ottenere i minuti rimanenti
all'arrivo di un mezzo pubblico ma altri si sono scontrati con quelle di Trenitalia \cite{trenitaliashock}

\subsection{Dettagli su ATM}
Le API di ATM sono usate per \href{https://giromilano.atm.it}{Giromilano} e l'unico modo per comprenderne il funzionamento è utilizzare
gli strumenti del browser per leggere le richieste e le risposte che vengono scambiate tra client e server durante la ricerca di un
percorso o di informazioni su una fermata specifica.\\
In particolare per ottenere i minuti rimanenti all'arrivo di un mezzo ad una specifica fermata, è necessario effettuare una richiesta
POST a\\ \url{https://giromilano.atm.it/proxy.ashx} e nel body inserire\\ \texttt{url=tpPortal/geodata/pois/stops/xxxxx}, dove a
\texttt{xxxxx} va sostituito il numero della fermata, leggibile (a volte) nel mondo reale sulla pensilina o sul cartello o direttamente
dal sito.\\
Per ottenere una risposta è inoltre necessario inviare l'header\\ \texttt{Content-Type: application/x-www-form-urlencoded}.\\
A questo punto sorgono alcuni problemi:
\begin{itemize}
  \item Le risposte contengono un po' di tutto e per via degli avvisi comprensivi di tag HTML possono anche essere molto lunghe, anche
        decine di kb.
  \item In caso di richiesta non corretta, ad esempio mancante di header Content-Type, non si avranno in risposta codici HTTP specifici
        ma sempre 200 (OK) in cui però la risposta è vuota.
\end{itemize}

A prescindere dalle questioni tecniche il problema è anche legale: non essendo API pubbliche l'utilizzo che se ne può fare all'esterno
dell'ambito aziendale è molto limitato. Può andare bene per utilizzi sperimentali personali, ma non si possono certamente incorporare
in prodotti da distribuire a qualsiasi titolo.\\
Poter risalire al ritardo accumulato da un mezzo pubblico (specialmente per treni con cadenza oraria o semioraria) senza dover navigare
siti e app ma direttamente lanciando un'occhiata a un angolo della stanza, come si farebbe con un orologio, sarebbe una comodità in più
per i pendolari che al momento non è attuabile su larga scala da terzi e le aziende di trasporti non sembrano interessate a proporre
prodotti simili.



\bibliographystyle{plain}
\bibliography{Biblio}
%\addcontentsline{toc}{chapter}{Bibliografia}

\end{document}
