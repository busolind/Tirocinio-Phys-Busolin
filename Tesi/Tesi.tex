% scopiazzato dal template di Matteo Longeri (grazie!)
%%%%%%%%%%%%%%%%%%%%%%%%%%%%%%%%%%%%%%%%%%%%%%%%%%%%%%
\documentclass[12pt,a4paper]{report}
% o article, book, ...



%%%%%%%%%%%%%%%%%%%%%%%%%%%%%%%%%%%%%%%%%%%%%%%%%%%%%%
% packages...
\usepackage[utf8]{inputenc}
\usepackage[english,italian]{babel}
\usepackage[hyphens]{url}

% Per generare il file PDF aderente alle specifiche PDF/A-1b. Verificarne poi la validità.
%\usepackage[a-1b]{pdfx}

\usepackage{hyperref}
\usepackage{graphicx}

\usepackage{underscore} % !!!
\usepackage{hyperref}
%\usepackage{bera}% optional: just to have a nice mono-spaced font
\usepackage{listings}
\usepackage{xcolor}

\colorlet{punct}{red!60!black}
\definecolor{background}{HTML}{EEEEEE}
\definecolor{delim}{RGB}{20,105,176}
\colorlet{numb}{magenta!60!black}

\lstdefinelanguage{json}{
    basicstyle=\normalfont\ttfamily,
    numbers=left,
    numberstyle=\scriptsize,
    stepnumber=1,
    numbersep=8pt,
    showstringspaces=false,
    breaklines=true,
    frame=lines,
    backgroundcolor=\color{background},
    literate=
     *{0}{{{\color{numb}0}}}{1}
      {1}{{{\color{numb}1}}}{1}
      {2}{{{\color{numb}2}}}{1}
      {3}{{{\color{numb}3}}}{1}
      {4}{{{\color{numb}4}}}{1}
      {5}{{{\color{numb}5}}}{1}
      {6}{{{\color{numb}6}}}{1}
      {7}{{{\color{numb}7}}}{1}
      {8}{{{\color{numb}8}}}{1}
      {9}{{{\color{numb}9}}}{1}
      {:}{{{\color{punct}{:}}}}{1}
      {,}{{{\color{punct}{,}}}}{1}
      {\{}{{{\color{delim}{\{}}}}{1}
      {\}}{{{\color{delim}{\}}}}}{1}
      {[}{{{\color{delim}{[}}}}{1}
      {]}{{{\color{delim}{]}}}}{1},
}


% Per inserire testo a caso in attesa di realizzare i capitoli
\usepackage{lipsum}



%%%%%%%%%%%%%%%%%%%%%%%%%%%%%%%%%%%%%%%%%%%%%%%%%%%%%
\begin{document}

% Frontespizio
\begin{titlepage}
  \begin{center}
    \includegraphics[width=\textwidth]{Logo.jpg}\\
    {\large{\bf Corso di Laurea in Informatica}}
  \end{center}
  \vspace{12mm}
  \begin{center}
    {\huge{\bf CONVERSIONE DI}}\\
    \vspace{4mm}
    {\huge{\bf STRUMENTI VINTAGE}}\\
    \vspace{4mm}
    {\huge{\bf PER LA DATA}}\\
    \vspace{4mm}
    {\huge{\bf PHYSICALIZATION}}\\
  \end{center}
  \vspace{12mm}
  \begin{flushleft}
    {\large{\bf Relatore:}}
    {\large{Andrea Trentini}}\\
    %\vspace{4mm}
    %{\large{\bf Correlatore:}}
    %{\large{...}}\\
  \end{flushleft}
  \vspace{12mm}
  \begin{flushright}
    {\large{\bf Tesi di Laurea di:}}
    {\large{Davide Busolin}}\\
    {\large{\bf Matr. 930814}}\\
  \end{flushright}
  \vspace{4mm}
  \begin{center}
    {\large{\bf Anno Accademico 2020/2021}}
  \end{center}
\end{titlepage}


\tableofcontents


% REPORT
%\part
%	\chapter
%		\section
%			\subsection
%				\paragraph
%					\subparagraph

% o sections (dipende dal documentclass)
\chapter{Introduzione}
TCP/IP over Avian Carriers\cite{waitzman1990standard}

\chapter{Attori coinvolti}
Prima di parlare del funzionamento dell'oggetto è il caso di soffermarsi su chi interagisce con esso:
\subsection*{Sviluppatore}
È in grado di apportare modifiche al codice sorgente. Può aggiungere, rimuovere e  modificare funzionalità del programma
e apportare modifiche dirette ai file di configurazione.
\subsection*{Utente esperto}
Conosce le API ed è in grado di comprendere il formato dei dati che restituiscono. Gli è sufficiente un'interfaccia anche spartana
per selezionare la sorgente dei dati e il campo specifico che vuole rappresentato.
\subsection*{Utente inesperto}
Non conosce le API né il formato JSON: ha bisogno di un prodotto già pronto che richieda la minima configurazione possibile e questa
deve essere particolarmente intuitiva, ad esempio una piccola interfaccia web per scegliere la rete WiFi e inserirne la password
al primo avvio. La sorgente dei dati deve essere preconfigurata e se ne viene resa disponibile più di una la scelta deve essere molto
semplice, possibilmente tramite interazione fisica con il dispositivo.

\chapter{Funzionamento}

\section{All'avvio}
Per prima cosa contestualmente all'accensione, mediante WiFiManager, viene effettuato un tentativo di connessione all'ultima rete WiFi
utilizzata, di cui sono state salvate le credenziali. Se non è mai stata effettuata la connessione ad alcuna rete WiFi o se
quella salvata non è disponibile, viene attivato un Access Point che tramite Captive Portal (lo stesso che viene usato su reti pubbliche
o aziendali per richiedere l'autenticazione) chiede di selezionare una delle reti WiFi vicine e di inserirne la password.\\

\vspace{4mm}
{[SCREENSHOT]}\\
\vspace{4mm}

Successivamente viene tentata la connessione. Se non ha successo viene riproposto lo stesso procedimento, altrimenti viene disattivato
l'Access Point temporaneo e il programma prosegue.

\pagebreak
\section{Modalità operative}

\subsection{Modalità API}
Questa modalità di operazione si basa sull'esecuzione ad intervalli di tempo configurabili di richieste HTTP o HTTPS ad API REST
e sulla conversione del risultato in una tensione di uscita.
La configurazione è salvata nella memoria flash del microcontrollore
in formato JSON, esempio:

\begin{lstlisting}[language=json,firstnumber=1]
{
  "apiUrl": "http://api.coindesk.com/v1/bpi/currentprice/USD.json",
  "filterJSON": "{bpi:{USD:{rate_float:true}}}",
  "path": "bpi/USD/rate_float",
  "min_value": 58700,
  "max_value": 58900,
  "min_pwm": 0,
  "max_pwm": 1023,
  "request_interval_ms": 15000
}
\end{lstlisting}


\noindent Dove:
\begin{itemize}
  \item \texttt{apiUrl} è l'URL della risorsa (stringa)
  \item \texttt{post_payload} in caso di richiesta POST contiene la stringa da inviare nel body
  \item \texttt{filterJSON} è un documento JSON che contiene \texttt{true} come placeholder del campo che si vuole considerare dalla risposta (stringa/oggetto)
  \item \texttt{path} è una stringa che contiene il "percorso" del campo che si vuole rappresentare fisicamente (campo: float, path: stringa)
  \item \texttt{min_value} indica il valore minimo della scala per rappresentare il valore (float)
  \item \texttt{max_value} indica il valore massimo (float)
  \item \texttt{min_pwm} indica il valore minimo di uscita della PWM da mappare al valore restituito dalle API (int)
  \item \texttt{max_pwm} indica il valore massimo [su ESP il duty cycle massimo equivale a 1023] (int)
  \item \texttt{request_interval_ms} indica l'intervallo di tempo in millisecondi che intercorre tra le richieste alla risorsa (int)
\end{itemize}

La configurazione può essere alterata tramite una pagina web accessibile all'indirizzo IP nella rete locale del microcontrollore o il suo
hostname.

\vspace{4mm}
{[SCREENSHOT PAGINA WEB]}

\vspace{4mm}
I form sono precompilati con la configurazione in esecuzione (\emph{running-conf}) ed è presente una sezione che riporta la configurazione
salvata nella flash (\emph{saved-conf}).\\
È possibile inoltre inviare un JSON di configurazione tramite MQTT al topic \texttt{Phys/setFromJSON}. I campi possono essere modificati
singolarmente ai topic:
\begin{itemize}
  \item \texttt{Phys/setApiUrl}
        % ToDo: setPostPayload
  \item \texttt{Phys/setFilterJson}
  \item \texttt{Phys/setPath}
  \item \texttt{Phys/setMinValue}
  \item \texttt{Phys/setMaxValue}
  \item \texttt{Phys/setMinPwm}
  \item \texttt{Phys/setMaxPwm}
  \item \texttt{Phys/setRequestIntervalMs}
\end{itemize}

Le modifiche non sono automaticamente salvate nella memoria flash ma ciò deve essere richiesto esplicitamente. Nei form di configurazione
è presente una checkbox da spuntare in caso si voglia che questo avvenga.


\subsection{Modalità MQTT}
Non ancora implementata, ma l'idea è che consenta di visualizzare un valore ricevuto direttamente tramite sottoscrizione a un topic MQTT
invece che andarlo a reperire tramite richieste HTTP(S).


\chapter{Conclusioni}

\section{Reperibilità delle API}
Durante la ricerca di API rappresentabili mediante Data Physicalization è risultato evidente che molte di quelle più congeniali non
sono pubbliche e spesso sono anche di difficile utilizzo. L'esperienza personale è stata con quelle di ATM per ottenere i minuti rimanenti
all'arrivo di un mezzo pubblico ma altri si sono scontrati con quelle di Trenitalia \cite{trenitaliashock}

\subsection{Dettagli su ATM}
Le API di ATM sono usate per \href{https://giromilano.atm.it}{Giromilano} e l'unico modo per comprenderne il funzionamento è utilizzare
gli strumenti del browser per leggere le richieste e le risposte che vengono scambiate tra client e server durante la ricerca di un
percorso o di informazioni su una fermata specifica.\\
In particolare per ottenere i minuti rimanenti all'arrivo di un mezzo ad una specifica fermata, è necessario effettuare una richiesta
POST a\\ \url{https://giromilano.atm.it/proxy.ashx} e nel body inserire\\ \texttt{url=tpPortal/geodata/pois/stops/xxxxx}, dove a
\texttt{xxxxx} va sostituito il numero della fermata, leggibile (a volte) nel mondo reale sulla pensilina o sul cartello o direttamente
dal sito.\\
Per ottenere una risposta è inoltre necessario inviare l'header\\ \texttt{Content-Type: application/x-www-form-urlencoded}.\\
A questo punto sorgono alcuni problemi:
\begin{itemize}
  \item Le risposte contengono un po' di tutto e per via degli avvisi comprensivi di tag HTML possono anche essere molto lunghe, anche
        decine di kb.
  \item In caso di richiesta non corretta, ad esempio mancante di header Content-Type, non si avranno in risposta codici HTTP specifici
        ma sempre 200 (OK) in cui però la risposta è vuota.
\end{itemize}

A prescindere dalle questioni tecniche il problema è anche legale: non essendo API pubbliche l'utilizzo che se ne può fare all'esterno
dell'ambito aziendale è molto limitato. Può andare bene per utilizzi sperimentali personali, ma non si possono certamente incorporare
in prodotti da distribuire a qualsiasi titolo.\\
Poter risalire al ritardo accumulato da un mezzo pubblico (specialmente per treni con cadenza oraria o semioraria) senza dover navigare
siti e app ma direttamente lanciando un'occhiata a un angolo della stanza, come si farebbe con un orologio, sarebbe una comodità in più
per i pendolari che al momento non è attuabile su larga scala da terzi e le aziende di trasporti non sembrano interessate a proporre
prodotti simili.



\bibliographystyle{plain}
\bibliography{Biblio}
%\addcontentsline{toc}{chapter}{Bibliografia}

\end{document}
