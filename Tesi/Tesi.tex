% scopiazzato dal template di Matteo Longeri (grazie!)
%%%%%%%%%%%%%%%%%%%%%%%%%%%%%%%%%%%%%%%%%%%%%%%%%%%%%%
\documentclass[12pt,a4paper]{report}
% o article, book, ...



%%%%%%%%%%%%%%%%%%%%%%%%%%%%%%%%%%%%%%%%%%%%%%%%%%%%%%
% packages...
\usepackage[utf8]{inputenc}
\usepackage[english,italian]{babel}
\usepackage[hyphens]{url}

% Per generare il file PDF aderente alle specifiche PDF/A-1b. Verificarne poi la validità.
%\usepackage[a-1b]{pdfx}

\usepackage{hyperref}
\usepackage{graphicx}


% Per inserire testo a caso in attesa di realizzare i capitoli
\usepackage{lipsum}



%%%%%%%%%%%%%%%%%%%%%%%%%%%%%%%%%%%%%%%%%%%%%%%%%%%%%
\begin{document}

% Frontespizio
\begin{titlepage}
  \begin{center}
    \includegraphics[width=\textwidth]{Logo.jpg}\\
    {\large{\bf Corso di Laurea in Informatica}}
  \end{center}
  \vspace{12mm}
  \begin{center}
    {\huge{\bf CONVERSIONE DI}}\\
    \vspace{4mm}
    {\huge{\bf STRUMENTI VINTAGE}}\\
    \vspace{4mm}
    {\huge{\bf PER LA DATA}}\\
    \vspace{4mm}
    {\huge{\bf PHYSICALIZATION}}\\
  \end{center}
  \vspace{12mm}
  \begin{flushleft}
    {\large{\bf Relatore:}}
    {\large{Andrea Trentini}}\\
    %\vspace{4mm}
    %{\large{\bf Correlatore:}}
    %{\large{...}}\\
  \end{flushleft}
  \vspace{12mm}
  \begin{flushright}
    {\large{\bf Tesi di Laurea di:}}
    {\large{Davide Busolin}}\\
    {\large{\bf Matr. 930814}}\\
  \end{flushright}
  \vspace{4mm}
  \begin{center}
    {\large{\bf Anno Accademico 2020/2021}}
  \end{center}
\end{titlepage}


\tableofcontents


% REPORT
%\part
%	\chapter
%		\section
%			\subsection
%				\paragraph
%					\subparagraph

% o sections (dipende dal documentclass)
\chapter{Introduzione}
TCP/IP over Avian Carriers\cite{waitzman1990standard}

\chapter{Funzionamento}
\section{Attori coinvolti}
Prima di parlare del funzionamento dell'oggetto è il caso di soffermarsi su chi interagisce con esso
\subsection*{Sviluppatore}
È in grado di apportare modifiche al codice sorgente. Può aggiungere, rimuovere e modificare funzionalità del programma
e apportare modifiche dirette ai file di configurazione.
\subsection*{Utente esperto}
Conosce le API ed è in grado di comprendere il formato dei dati che restituiscono. Gli è sufficiente un'interfaccia anche spartana
per selezionare la sorgente dei dati e il campo specifico che vuole rappresentato.
\subsection*{Utente inesperto}
Non conosce le API né il formato JSON: ha bisogno di un prodotto già pronto che richieda la minima configurazione possibile e questa
deve essere particolarmente intuitiva, ad esempio una piccola interfaccia web per scegliere la rete WiFi e inserirne la password
al primo avvio. La sorgente dei dati deve essere preconfigurata e se ne viene resa disponibile più di una la scelta deve essere molto
semplice, possibilmente tramite interazione fisica con il dispositivo.
\section{Modalità API}

\section{Modalità MQTT}
ToDo
\chapter{Cap3}
\chapter{Conclusioni}


\bibliographystyle{plain}
\bibliography{Biblio}
%\addcontentsline{toc}{chapter}{Bibliografia}

\end{document}
